\documentclass[a4paper,10pt]{article}
\usepackage[utf8]{inputenc}
\usepackage[T1]{fontenc}
\usepackage{lmodern}
\usepackage[ngerman]{babel}
\usepackage{amsmath}
\usepackage{fullpage}
\usepackage{listings}
\usepackage{amssymb}
\usepackage{newclude}
\usepackage{multirow}
\usepackage{longtable}
%\pagestyle{empty}


\lstset{ %
  %backgroundcolor=\color{white},   % choose the background color; you must add \usepackage{color} or \usepackage{xcolor}
  %basicstyle=\footnotesize,        % the size of the fonts that are used for the code
  %breakatwhitespace=false,         % sets if automatic breaks should only happen at whitespace
  breaklines=true,                 % sets automatic line breaking
  %captionpos=b,                    % sets the caption-position to bottom
  commentstyle=\bf,    % comment style
  %deletekeywords={...},            % if you want to delete keywords from the given language
  %escapeinside={\%*}{*)},          % if you want to add LaTeX within your code
  %extendedchars=true,              % lets you use non-ASCII characters; for 8-bits encodings only, does not work with UTF-8
  frame=single,                    % adds a frame around the code
  %keepspaces=true,                 % keeps spaces in text, useful for keeping indentation of code (possibly needs columns=flexible)
  %keywordstyle=\color{blue},       % keyword style
  language=Java,                 % the language of the code
  %otherkeywords={*,...},            % if you want to add more keywords to the set
  numbers=left,                    % where to put the line-numbers; possible values are (none, left, right)
  %numbersep=5pt,                   % how far the line-numbers are from the code
  %numberstyle=\tiny\color{mygray}, % the style that is used for the line-numbers
  %rulecolor=\color{black},         % if not set, the frame-color may be changed on line-breaks within not-black text (e.g. comments (green here))
  %showspaces=false,                % show spaces everywhere adding particular underscores; it overrides 'showstringspaces'
  %showstringspaces=false,          % underline spaces within strings only
  %showtabs=false,                  % show tabs within strings adding particular underscores
  %stepnumber=2,                    % the step between two line-numbers. If it's 1, each line will be numbered
  %stringstyle=\color{mymauve},     % string literal style
  tabsize=4,                       % sets default tabsize to 2 spaces
  %title=\lstname                   % show the filename of files included with \lstinputlisting; also try caption instead of title
}


\begin{document}
\hfill \\
Adam Shafei \hfill Len Williamson \hfill Steffen Pegenau
\section*{3. Übungszettel in EiSE \-- Gruppe 073 \-- WiSe 2015/16}
\subsection*{Aufgabe 1}
\subsubsection*{a) Funktionale und nicht funktionale Anforderungen}
\begin{longtable}{|p{0.47\linewidth}|p{0.47\linewidth}|}
\hline
\multicolumn{2}{|c|}{\textbf{Anforderungen}} \\
\hline
\textbf{funktional} & \textbf{nicht funktional} \\
\hline
\hline
\multicolumn{2}{|c|}{\textbf{Allgemein}} \\
\hline
Design \& Bedienung an Endgerät angepasst & Hauptoptionen benutzerfreundlich im Hauptmenü erreichbar \\
\hline
Login des Nutzers erfolgt mit Benutzernamen und PIN & Ansprechendes Design \\
\hline
Auto-Logout des Nutzers nach 30 Minuten Inaktivität & schnelle Antwortzeiten \\
\hline
Nutzer kann sich selbst ausloggen & Anwendung sicher vor unerlaubtem Zugriff und allen Angriffen\\
\hline
Alle Eingaben und Ansichten sollen auch für Nutzer mit Sehbehinderung nutzbar sein & Dauerhafte Erreichbarkeit \\
\hline
Nutzer wird zu Beginn der Sitzung über etwaige Behinderung befragt & \\
\hline
\hline
\multicolumn{2}{|c|}{\textbf{Überweisung}} \\
\hline
Nutzer kann Standard- oder Terminüberweisungen sowie Daueraufträge tätigen & Bedienerfreundliche Eingabe des Datums bei Terminüberweisungen \\
\hline
Ermittlung des Überweisungsziels mit IBAN oder Kontonummer/BLZ & \\
\hline
Der Nutzer kann den Geldbetrag angeben & \\
\hline
Der Nutzer kann Verwendungszweck/Kundenreferenznummer angeben & \\
\hline
Abfrage des exklusiven Überweisungstyps (Standard- oder Terminüberweisungen, Dauerauftrag) am Ende des Formulars (exklusive Auswahl) & \\
\hline
Validitätsprüfung aller Eingabe nach Abschicken des Formulars durch Nutzer & \\
\hline
Schlägt Validitätsprüfung fehl, wird Nutzer auf fehlende/fehlerhafte Eingaben aufmerksam gemacht & \\
\hline
Ist die Validitätsprüfung erfolgreich, bekommt Nutzer Zusammenfassung seiner Eingaben & \\
\hline
Abfrage der TAN (abhängig von TAN-Einstellung) & \\
\hline
Ist TAN korrekt wird Transaktion ausgeführt & \\
\hline
Nach Ausführung der Transaktion wird Nutzer gefragt, ob er weitere Überweisung tätigen will oder zurück zum Hauptmenü will & \\
\hline
Wurde die falsche TAN eingegeben, wird der Nutzer nach TAN-Verfahren zur Eingabe einer anderen, bestimmten TAN aufgefordert bis Prüfung erfolgreich oder der Nutzer die Überweisung abbricht & \\
\hline
\hline
\multicolumn{2}{|c|}{\textbf{TAN-Einstellungen}} \\
\hline
Nutzer kann das verwendete TAN-Verfahren (mTAN, ChipTAN, TAN-Liste) ändern & \\
\hline
Bei mTAN wird dem Nutzer die TAN mit Zusammenfassung der Überweisung per SMS ans Handy geschickt & \\
\hline
Zum Wechsel zu mTAN muss der Nutzer seine Handynummer hinterlegen & \\
\hline
Bei ChipTAN erhält der Nutzer mit der Überweisungszusammenfassung einen Code, den er mit einer Chip-Karte ins Lesegerät eingibt. Das Lesegerät berechnet anschließend die TAN & \\
\hline
Nutzer kann neue TAN-Liste in den Einstellungen mit einer alten TAN anfordern & \\
\hline
Fordert der Nutzer eine neue TAN-Liste erfolgreich an werden alle aktiven TANs der alten Liste gesperrt. & \\
\hline
Sind nur noch 10 TANs einer Liste übrig, wird automatisch eine neue TAN-Liste per Post versandt & \\
\hline
Wird eine TAN einer neuen Liste genutzt, werden alle TANs der alten Liste gesperrt & \\
\hline
Kunde kann neue TAN-Liste telefonisch bei Service-Mitarbeiter anfordern, wenn alte Liste unauffindbar & \\
\hline
Service-Mitarbeiter haben auf alle Funktionalitäten des Kunden Zugriff &  \\
\hline
\hline
\multicolumn{2}{|c|}{\textbf{Depot einsehen/Kontoauszüge}} \\
\hline
Kunden-Depot-Ansicht 1: Liste aller Transaktionen der letzten 30 Tage sowie Kontostand & Zeitraum gut ersichtlich\\
\hline
Kunden-Depot-Ansicht 2: Liste aller Transaktion sowie Kontostand in einem frei wählbaren Zeitraum & Zeitraum leicht veränderbar\\
\hline
\hline
\end{longtable}

\subsubsection*{b) Fragen zur Umsetzung nicht funktionaler Anforderungen}
\begin{tabular}{|p{0.37\linewidth}|p{0.57\linewidth}|}
\hline
\textbf{Nicht funktionale Anforderung} & \textbf{Frage} \\
\hline
Allgemein \-- Hauptoptionen benutzerfreundlich im Hauptmenü erreichbar &  Was kritisieren Kunden an der Bedienung des bestehenden Systems? \\
\hline
Allgemein \-- Ansprechendes Design & Welche Design-Richtlinien gibt es im Unternehmen? Wie groß sind die Freiheiten bei der Entwicklung der Oberfläche? \\
\hline
Allgemein \-- Schnelle Antwortzeiten & Was heißt ``schnell''? Werden bestimmte Antwortzeiten garantiert? Wird der Zugriff durch Kunden weltweit, kontinental oder national erfolgen?  Gibt es besondere Peaks in den Zugriffszahlen? Wie sehen die Wachstumszahlen beim Online-Banking aus? Wie sieht die Unternehmensstrategie bezüglich Online-Banking aus? \\
\hline
Allgemein \-- Anwendung sicher vor unerlaubtem Zugriff und allen Angriffen & Gibt es besondere Sicherheitsrichtlinien des Unternehmens? Gibt es Erfahrungen mit Sicherheitsbrüchen? (Wann) Soll ein externer Code-Review erfolgen? \\
\hline
Allgemein \-- Dauerhafte Erreichbarkeit & Was heißt dauerhaft? Gibt es rechtliche oder unternehmensinterne Regelungen? In welchem Umfang soll das Online-Banking über mehrere Server skalieren? \\
\hline
\hline
Überweisung \-- Bedienerfreundliche Eingabe des Datums bei Terminüberweisungen & Gibt es Vorstellungen was bedienerfreundlich heißt bzw. was unbedingt vermieden werden sollte? Wieder: Gibt es Design-Richtlinien des Unternehmens? \\
\hline
\hline
Depot einsehen/Kontoauszüge \-- Zeitraum gut ersichtlich & Wiederholung: Gibt es Vorstellungen was ``gut ersichtlich'' heißt bzw. was unbedingt vermieden werden sollte? Gibt es Design-Richtlinien des Unternehmens? \\
\hline
Depot einsehen/Kontoauszüge \-- Zeitraum leicht veränderbar & Wiederholung: Gibt es Vorstellungen was ``leicht veränderbar'' heißt bzw. was unbedingt vermieden werden sollte? Gibt es Design-Richtlinien des Unternehmens?\\
\hline
\end{tabular}

\subsection*{Aufgabe 2}
\subsection*{Aufgabe 3}
\subsubsection*{a) Use Case ``Funktionalität einer Überweisung''}
\begin{tabular}{|p{0.37\linewidth}|p{0.57\linewidth}|}
\hline
\textbf{Use Case Abschnitt} & \textbf{Zweck} \\
\hline
Use Case Name & \\
\hline
Scope & \\
\hline
Level & \\
\hline
Primary Actor & \\
\hline
Stakeholders and Interests & \\
\hline
Preconditions & \\
\hline
Minimal guarantees & \\
\hline
Success Guarantee & \\
\hline
Main Success Scenario & \\
\hline
Extensions & \\
\hline
Special Requirements & \\
\hline
Technology and Data Variation List & \\
\hline
Frequency of Occurrence & \\
\hline
Miscellaneous & \\
\hline
\hline
\end{tabular}

\subsubsection*{b) Use Case ``Neue TAN-Liste Versenden''}
\begin{tabular}{|p{0.37\linewidth}|p{0.57\linewidth}|}
\hline
\textbf{Use Case Abschnitt} & \textbf{Zweck} \\
\hline
Use Case Name & \\
\hline
Scope & \\
\hline
Level & \\
\hline
Primary Actor & \\
\hline
Stakeholders and Interests & \\
\hline
Preconditions & \\
\hline
Minimal guarantees & \\
\hline
Success Guarantee & \\
\hline
Main Success Scenario & \\
\hline
Extensions & \\
\hline
Special Requirements & \\
\hline
Technology and Data Variation List & \\
\hline
Frequency of Occurrence & \\
\hline
Miscellaneous & \\
\hline
\hline
\end{tabular}
\end{document}
