\documentclass[a4paper,10pt]{article}
\usepackage[utf8]{inputenc}
\usepackage[T1]{fontenc}
\usepackage{lmodern}
\usepackage[ngerman]{babel}
\usepackage{amsmath}
\usepackage{fullpage}
\usepackage{listings}
\usepackage{amssymb}
\usepackage{newclude}
\usepackage{multirow}
\pagestyle{empty}


\lstset{ %
  %backgroundcolor=\color{white},   % choose the background color; you must add \usepackage{color} or \usepackage{xcolor}
  %basicstyle=\footnotesize,        % the size of the fonts that are used for the code
  %breakatwhitespace=false,         % sets if automatic breaks should only happen at whitespace
  breaklines=true,                 % sets automatic line breaking
  %captionpos=b,                    % sets the caption-position to bottom
  commentstyle=\bf,    % comment style
  %deletekeywords={...},            % if you want to delete keywords from the given language
  %escapeinside={\%*}{*)},          % if you want to add LaTeX within your code
  %extendedchars=true,              % lets you use non-ASCII characters; for 8-bits encodings only, does not work with UTF-8
  frame=single,                    % adds a frame around the code
  %keepspaces=true,                 % keeps spaces in text, useful for keeping indentation of code (possibly needs columns=flexible)
  %keywordstyle=\color{blue},       % keyword style
  language=Java,                 % the language of the code
  %otherkeywords={*,...},            % if you want to add more keywords to the set
  numbers=left,                    % where to put the line-numbers; possible values are (none, left, right)
  %numbersep=5pt,                   % how far the line-numbers are from the code
  %numberstyle=\tiny\color{mygray}, % the style that is used for the line-numbers
  %rulecolor=\color{black},         % if not set, the frame-color may be changed on line-breaks within not-black text (e.g. comments (green here))
  %showspaces=false,                % show spaces everywhere adding particular underscores; it overrides 'showstringspaces'
  %showstringspaces=false,          % underline spaces within strings only
  %showtabs=false,                  % show tabs within strings adding particular underscores
  %stepnumber=2,                    % the step between two line-numbers. If it's 1, each line will be numbered
  %stringstyle=\color{mymauve},     % string literal style
  tabsize=4,                       % sets default tabsize to 2 spaces
  %title=\lstname                   % show the filename of files included with \lstinputlisting; also try caption instead of title
}


\begin{document}
\hfill \\
Adam Shafei \hfill Len Williamson \hfill Steffen Pegenau
\section*{1. Übungszettel in Software Engineering}

\subsection*{Tatsächlich benötigte Zeit}
\subsubsection*{Übersicht}
\begin{tabular}{|l|l|l|l|l|}
 \hline
 \textbf{Klasse} & \textbf{Schätzung [h]} & \textbf{Realität [h]} & \textbf{Differenz  [h]}& \textbf{Realität/Schätzung} \\
 \hline
 GUIConsole & 6,25 & 13 & 6,75 & 2,08 \\
 \hline
 DotsNBoxesEngine & 7,5 & 10 & 2,5 & 1,33 \\
 \hline
 Player & 0,5 & 0,5 & 0 & 1 \\
 \hline
 \hline
 \textbf{Summe} & 14,25 & 23,5 & 9,25 & 1,62\\
 \hline
 \end{tabular} \hfill \\[0.5cm]
 Wir benötigten ca. 62\% mehr Zeit als prognostiziert. Hauptsächlich lag das daran, dass wir unsere Fähigkeiten über- und Detailfragen unterschätzt haben. \\
 Im Laufe der Bearbeitung haben wir gelernt, die Programmierung von einzelnen Methoden besser aufzuteilen, indem wir Schnittstellen definieren und die Ergebnisse mit Git schnell zusammenzuführen. Außerdem hilfreich waren Diskussionen und das gegenseitige Überprüfen von Code. \\
 Durch diese Lernfortschritte, die Erfahrung und die Erfassung der Diskrepanz, hoffen wir, Zeitabschätzungen in Zukunft realistischer geben zu können.
\subsubsection*{Klasse GUIConsole}
\begin{tabular}{|l|l|l|p{0.4\linewidth}|}
 \hline
 \textbf{Aufgabe} & \textbf{Schätzung [h]} & \textbf{Realität [h]} & \textbf{Begründung} \\
 \hline
 Eingabe & 0,75 & 4,5 & Zunächst entstanden viele Redundanzen, sodass es schwierig war, Optimierungen (an vielen Stellen) umzusetzen. Außerdem machte die Stream-Natur des Inputs Probleme bei der Prüfung (Öffnen und Schließen des Inputs sorgte für Endlosschleife.\\
 \hline
 Darstellung Spielfeld & 1 & 1 & \\
 \hline
 Anzeige & 0,5 & 1,5 & Doppeltes Hochzählen um aktuellen Spieler zu ermitteln führte zu eine Null-Pointer-Exception, die nicht direkt gefunden wurde. \\
 \hline
 Tests & 2,5 & 3 & Schnittstellen zwischen Klassen waren zunächst zu undeutlich abgestimmt. Dadurch war es schwierig brauchbare Tests zu schreiben. In Folge wurde viel manuell getestet, was viel Zeit in Anspruch nahm. \\
 \hline
 Fehlerkorrekturen & 1,5 & 3 & Es gab unerwartet viel zu optimieren. \\
 \hline
 \hline
 \textbf{Summe} & 6,25 & 13 & Abweichung: 6,75h\\
 \hline
\end{tabular}
\subsubsection*{Klasse DotsNBoxesEngine}
\begin{tabular}{|l|l|l|p{0.4\linewidth}|}
 \hline
 \textbf{Aufgabe} & \textbf{Schätzung [h]} & \textbf{Realität [h]} & \textbf{Begründung} \\
 \hline
 Datenstruktur Karte & 1 & 1 & \\
 Zug durchführen & 2 & 4 & Es hat viel Zeit gekostet zu ermitteln, welcher Spieler als nächstes an der Reihe ist.\\
 Punkteverwaltung & 0,5 & 0,5 & \\
  Tests & 2,5 & 3 & siehe Klasse GUIConsole \\
 \hline
 Fehlerkorrekturen & 1,5 & 1,5 &  \\
 \hline
 \hline
 \textbf{Summe} & 7,5 & 10 & Abweichung: 2,5h\\
 \hline
\end{tabular}
\subsubsection*{Klasse Player}
Genau wie erwartet. Wenig Aufwand.

\subsubsection*{Teilaufgaben und Schätzungen}
\begin{tabular}{|p{0.35\textwidth}|p{0.44\textwidth}|p{0.15\textwidth}|}
 \hline
 \textbf{Aufgabe} & \textbf{Begründung/Kommentar} & \textbf{Schätzung [h]} \\
 \hline


 \textbf{Klasse GUIConsole} & Starten, Einstellen, Spielen & \textbf{SUMME: 6,25}\\
 \hline
 \-- Eingabe & \multirow{5}{0.4\textwidth}{Einarbeitung in Eingabemethoden auf der Konsole Konsole, gemäß Erfahrung geringe Komplexität, schnell umsetzbar} & \multirow{5}{0.15\textwidth}{0,75} \\
 \-- \-- Spieleranzahl & &\\
 \-- \-- Spielfelddimensionen & &\\
 \-- \-- Spielernamen & &\\
 \-- \-- Zug (Spieler x setzt Wand y) & &\\
 \hline
 \-- Darstellung Spielfeld & Darstellung soll Datenstruktur der Karte interpretieren, sodass eine Anpassung der Darstellung möglich ist, ohne die Datenstruktur grundsätzlich zu ändern. Darstellung muss sich bei großen Feldern anpassen. & 1 \\
 \hline
 \-- Anzeige & \multirow{4}{0.4\textwidth}{Anzeige zum richtigen Zeitpunkt, sonst keine besondere Komplexität. Inhalte werden über Methoden der jeweiligen Klassen gut abrufbar sein.} & \multirow{4}{0.15\textwidth}{0,5} \\
 \-- \-- Punkte & &\\
 \-- \-- Gewinner & &\\
 \-- \-- Zugzahl & &\\
 \hline
 \--  Tests schreiben & Schwierig, da möglichst viele Fälle (umfangreich) abgedeckt sein müssen und Übung fehlt. Erfordert viel Kreativität. & 2,5\\
 \hline
 \-- Fehlerkorrekturen \& Optimierungen & Etwas geringere Komplexität als die Tests. Nachdem die Tests laufen, sollten nicht mehr viele Korrekturen/Optimierungen nötig sein & 1,5\\
 \hline
 \multicolumn{3}{l}{} \\
 \hline
 \textbf{Klasse DotsNBoxesEngine} & Einhaltung der Regeln, Verwaltung des Spielstandes, Punkteberechnung & \textbf{SUMME: 7,5} \\
 \hline
 \-- Datenstruktur Karte (2D-Array) mit Initialisierung & Grundsätzliche geringe Komplexität, aber durch Zweidimensionalität wenig intuitiv & 1 \\
 \hline
 \-- Zug durchführen (boolean: true, wenn Zug erfolgreich) & \multirow{6}{0.4\textwidth}{Höchste Komplexität, da hier die komplette Spiellogik liegt (die zuerst verstanden sein muss) und viele Hilfsmethoden geschrieben werden müssen.} & 2\\
 \-- \-- Ist der Spieler an der Reihe? & &\\
 \-- \-- Ist Zug gültig? &  &\\
 \-- \-- Ist Käsekästchen voll? &  &\\
 \-- \-- Ist das Spiel beendet? &  &\\
 \-- \-- Wer ist als nächstes an der Reihe? &  &\\
 \hline
 \-- Punkteverwaltung & Geringe Komplexität, da es sich gut über getter und setter der Klasse Player umsetzen lässt & 0,5\\
 \hline
 \-- Tests schreiben & Wie in Klasse GUIConsole &  2,5\\
 \hline
 \-- Fehlerkorrekturen & Wie in Klasse GUIConsole & 1,5 \\
 \hline
  \multicolumn{3}{l}{} \\[0.01cm]
 \hline
 \textbf{Klasse Player} & Verwaltung Name, Punktestand & \textbf{SUMME: 0,5} \\
 \hline
 \-- Name setzen & \multirow{7}{0.4\textwidth}{Im Grunde nur getter- und setter-Methoden. Daher sehr geringe Komplexität} & \\
 \-- Name holen & &\\
 \-- Punkte setzen & &\\
 \-- Punkte holen & &\\
 \-- Punkte erhöhen um Wert x & &\\
 \-- Tests schreiben & &\\
 \-- Fehlerkorrekturen & &\\
 \hline
 \hline
 \textbf{GESAMTZEIT} & &  \textbf{14,25} \\
 \hline
\end{tabular}
\end{document}
